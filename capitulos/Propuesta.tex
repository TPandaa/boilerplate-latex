\chapter*{Propuesta de tesis}

\subsection*{Titulo}
El problema de la galería de arte y la variación de los k-modems.

\subsection*{Alumno}
Mendoza León Edgar

\subsection*{Asesor}
Carlos Hidalgo Toscano

\subsection*{Objetivo}
El objetivo de esta tesis es dar una introducción a una variación de el problema
de la galería de arte, el problema de los $k$-modems trata sobre visibilidad de
entre puntos dentro de un polígono simple, considerando que existen ciertos
puntos denominados puntos guardias, estos puntos especiales pueden
\textnormal{"ver"} a otros según su capacidad denotada por $k$, que es el numero
de aristas que atraviesa un segmento de linea que conecte el $k$ -módem y otro
punto.\\

Se pretende también dar una explicación de un algoritmo para encontrar
particiones de segmentos de lineas ortogonales que a su vez nos permiten dar
cotas para el numero de $k$ -modems que se necesitan para iluminar tanto
conjuntos de segmentos de lineas como poligonos ortogonales usando $k$ -modems.

\subsection*{Alcance}
Para la realización de la tesis nos apoyaremos en algunos compendios sobre el
problema de la galería de arte como los que escribieron Jorge Urrutia,
\text{O'Rouke}, Thomas Shermer y libros que traten sobre visibilidad en el
plano. Pretendemos también dar una revision de algunas variaciones o problemas
relacionados con el de la galería de arte usando diversos artículos sobre el
tema.

\subsection*{Justificación}
El tema de la galería de arte ilustra algunos conceptos matemáticos importantes
que se aprenden en la carrera de matemáticas tocando varia ramas de la
matemática y la computación, por ejemplo: la teoría de gráficas, conceptos
básicos de geometría plana, análisis de algoritmos, probabilidad, entre otros,
ademas que es un problema que es fácilmente comprendido por el publico en
general y que ademas tiene como producto muchas aplicaciones en áreas como la
investigación de operaciones, la robótica, la visión por computadoras y mas.

\subsection*{Bibliografia}


[l] Jose Aguilar. Iluminación de polígonos ortogonales y simples con birradares.
PhD Thesis. 2013.\\

[2] Oswin Aichholzer y col. Modem illumination of monotone polygons. En:
Computational Geometry 68 (mar, de 2018 ), pags. $101-118 .$ ISSN: $0925$ j.
comgeo $.2017 .05 .010 .$ URL: https://11nkinghub.elsevier.com/retrieve/pii/\\

[3] Mark de Berg y col. Computational Geometry: Algoritbms and Applications. en.
Springer Science \& Business Media, mar. de $2008 .$ ISBN: $978-3-540-77974-2$\\

[4] Adrian Bondy y U. S. R. Murty. Graph Theory: English. 2008. " ed. New York:
Springer, nov. de $2010 .$ ISBN: $978-1-84996-690-0$\\

[5] Frank Duque y Carlos Hidalgo-Toscano. An Upper Bound on the k-Modem
Illumination Problem. En: Int. $J$. Comput. Geom. Appl. 25.04 (dic. de 2015
), pags. $299-308$. ISSN: $0218-1959,1793-6357$\\

[6] Jiri Matousek. Geometric Discrepancy: An Mustrated Guide. en. Ed. por Jiri
Matousek. Algorithms and Combinatorics. Berlin Heidelberg: Springer-Verlag,
1999. ISBN: 978$3-540-65528-2 .$ DOI: $10.1007 / 978-3-642-03942-3 .$ URL:
https: //www. springer. com/gp/book/9783540655282.\\

[7]$\quad$ T. C. Shermer. Recent results in art galleries (geometry). En:
Proceedings of the IEEE 80.9 (sep. de 1992 ), pags. $1384-1399 .$ ISSN:
$1558-2256 .$ DOI: $10.1109 / 5.163407$\\

[8] Jorge Urrutia. Art Gallery and Illumination Problems. en. En: Handbook of
Computational Geometry. Elsevier, $1996,$ págs. $973-1027 .$ ISBN:
$978-0-444-82537-7 .$ DOL: $10.1016 / \mathrm{B} 978-044482537-7 / 50023-1 .$\\